\ifsolo
    ~

    \vspace{1cm}

    \begin{center}
        \textbf{\LARGE Free modules} \\[1em]
    \end{center}

    \vspace{1cm}

    \tableofcontents

    \vspace{1cm}

\else
    \chapter{Free modules}

    \minitoc
\fi
\thispagestyle{empty}

\section{Generators of a module}

\begin{dfn}[Submodule generated by a family of elements]
    Let M be an $R$-module, and $(x_i) _{i \in  I}$ be a family of elements of M. The submodule generated by $(x_i)_{i \in  I}$, denoted by $\langle (x_i)_{i \in  I}\rangle $, is the smallest submodule of $M$ containing all the elements of the family. \[\langle (x_i)_{i \in  I}\rangle = \bigcap_{\substack{N\subset M\\N\text{ submodule}\\ \forall i,x_i \in  N}}N = \left\{\sum_{i \in  I} a_ix_i, \qquad \text{almost all $a_i$ are zero}\right\}\]
\end{dfn}

\begin{dfn}[Generating family\index{generating family}\index{spanning family}]
We say that $(x_i)_{i \in  I}$ is a generating family or a spanning family of $M$ if $\langle (x_i)_{i \in  I}\rangle = M$.
\end{dfn}

\begin{dfn}
    We say that a family $(x_i)_{i \in  I}$ of elements of $M$ is free\index{free family} if \[\sum_{i \in  I} a_i x_i=0 \implies \forall i,a_i=0\]
\end{dfn}

\begin{dfn}
We say that $(x_i)_{i \in  I}$ is a basis of $M$ if it is both generating and free.
\end{dfn}

\begin{ex}
\begin{itemize}
    \item $ \left\{ 1 \right\} $ is a basis of $\Z$ as a $\Z$-module.
    \item $\Z / 2\Z$ contains no free family
\end{itemize}
\end{ex}

\section{Basis and free modules}

\begin{dfn}
A module $M$ is free if it has a basis
\end{dfn}

\begin{ex}
$\R^{(I)}$ is free with basis $(e_i)_{i \in I}=((\delta_{i,j})_{j \in  I})_{i \in  I}$.
\end{ex}

\begin{rem}
Not all modules are free !
\end{rem}

\begin{thm}
Let $I$ be a set, $M$ be an $R$-module. Then \[
\begin{array}{rrcl}
    & \Hom_R(R^{(I)},M) & \longrightarrow & M^I \\
    & f & \longmapsto & \displaystyle (f(e_i))_{i \in  I}
\end{array}
\] 
is an isomorphism.
\end{thm}

\begin{cor}
    Let $M$ be a free $R$-module. Then $M\simeq R^{(I)}$ for some $I$.
\end{cor}

\begin{proof}
Let $(x_i)_{i \in  I}$ be a basis of $M$, and define  \[
\begin{array}{rrcl}
    u:& R^{(I)} & \longrightarrow & M \\
    & e_i & \longmapsto & \displaystyle x_i
\end{array}
\] 
Then $u$ is surjective because $(x_i) _{i \in  I}$ spans $M$, and $x \in M$ is uniquely written $x=\sum a_ix_i = u((a_i)_{i \in I})$.
The map $u$ is also injective because $(x_i)_{i \in  I}$ is free (and $f(a)=f(b)$ makes a non-trivial zero linear combination of $(x_i)_{i \in  I}$, unless $a=b$).
\end{proof}

\begin{cor}
Let $M$ be free with basis  $(x_i)_{i \in  I}$. Let $N$ be an  $R$-module. The map  \[
\begin{array}{rrcl}
    & \Hom_R(M,N) & \longrightarrow & N^I \\
    & f & \longmapsto & \displaystyle f((x_i)_{i \in  I})
\end{array}
\]
is an isomorphism.
\end{cor}

\begin{thm}
Let $ M$ be a free module. Then all bases have the same cardinality.
\end{thm}

\begin{proof}
Let $(x_i)_{i \in  I}$ and $(y_j)_{j \in  J}$ be bases of $M$. Then \[
M \simeq R^{(I)}\simeq R^{(J)}
\]
thus $I$ and $J$ have the same cardinality, because if $\mathfrak m$ is a maximal ideal of $R$, then $K={R} / {\mathfrak m}$ is a field and $R^{(I)}/\mathfrak mR^{(I)}\simeq (R / \mathfrak m R)^{(I)}=K^{(I)}$, and the canonical map $f: R^{(I)} \longrightarrow  (\frac{R}{\mathfrak mR}^{(I)})$ is surjective with kernel $\mathfrak mR^{(I)}$, which induces an isomorphism $\bar f : R^{(I)} / \mathfrak mR^{(I)} \longrightarrow  (R /\mathfrak mR)^{(I)}=K^{(I)}$, therefore $K^{(I)}\simeq K^{J}$. The result is known for vector spaces.
\end{proof}

\begin{dfn}
    The cardinality of bases of a free module is called the rank of that module\index{module!rank}\index{rank (module)}.
\end{dfn}

\section{Finitely generated modules}

\begin{dfn}
    We say that $M$ is finitely generated if it is spanned by finitely many elements.\index{module!finitely generated}.
    Equivalently, there exists a surjective map \[
        R^n  \longrightarrow  M 
    \]
    for some $n$.
\end{dfn}

\begin{cor}
Quotients of finitely generated modules are finitely generated.
\end{cor}

\begin{rem}
    Let $M$ be a finitely generated module, and $N$ a submodule of $M$. Then $N$ \textbf{needs not be finitely generated}.

    For example, if $R$ is a non-noetherian ring, $R$ is an  $R$-module and it has a non-finitely generated ideal  $I$ which is also a  submodule.
\end{rem}

\subsection{Presentation of a module}

Let $M$ be an $R$-module,  $(x_i)_{i \in  I}$ a generating family. There is a surjective map $e_i R^{(I)} \longmapsto x_i \in  M$. This map is not enough to describe $M$.

A presentation\index{presentation} of $M$ is the data of an exact sequence  \[
    R^{(J)} \xrightarrow{u} R^{(I)}\xrightarrow \pi M \xrightarrow 0
\]
Since $M\simeq \coker u$, $M$ is entirely described by $u$.

\begin{dfn}
$M$ is finitely presented if there exists an exact sequence  \[
R^m \xrightarrow u R^n \xrightarrow\pi M \xrightarrow 0
\] 
Equivalently, $M$ is finitely generated and  $\ker (\pi)$ is also finitely generated.
\end{dfn}

\begin{rem}
    There exists modules that are finitely generated but not finitely presented\index{finitely presented module}\index{module!finitely presented}. For example, $M=R / I$ is finitely generated but $I$ can be taken to be non-finitely generated (in a non-noetherian ring).
\end{rem}


\section{Noetherian property}

\begin{dfn}
We say that an $R$-module is noetherian if all its submodules are finitely generated
\end{dfn}

\begin{rem}
    If $M$ is noetherian, then it is finitely generated. \index{module!noetherian} \index{noetherian module}
\end{rem}

\begin{ex}
$R$ as an  $R$-module is noetherian as a module  if and only if it is noetherian as a ring.
\end{ex}

\begin{prop}
If $ M$ is noetherian then all its submodules and quotients are also noetherian
\end{prop}


\begin{prop}
Let  \[
0 \longrightarrow M_1 \overset u\longrightarrow M_2\overset v\longrightarrow  M_3 \longrightarrow 0
\] 
be a short exact sequence. Then $M_2$ is noetherian  if and only if $M_1$ and  $M_3 $ also are.
\end{prop}

\begin{proof}
If $M_2$ is noetherian, then $M_1\simeq u(M_1)\subset M_2$ so $M_1$ is noetherian, and $M_3\simeq M_2 / \ker v$ which is also noetherian.
Conversely, assume $M_1$, $M_3$ are noetherian and let $N$ be a submodule of $M_2$. Then  $v(N)$ is a submodule of  $M_3$ so  $v(N)= \langle x_1, \cdots , x_i\rangle $. If  $y_1, \cdots, y_n$ are such that  $v(y_i)=x_i$, then  $u(M_1)\cap N$ is a submodule of  $u(M_1)\simeq M_1$ that is finitely generated. Let  $z_1, \cdots z_p$ be a generating family. Then  $\langle y_1, \cdots, y_n, z_1, \cdots, z_p\rangle =N$ because  $x \in  N$ can be written as $x = x' + \sum a_i y_i$ where $x' \in  \ker v=\im u$.
\end{proof}

\begin{cor}
If $M$ and  $N$ are  noetherian then so is $M\oplus N$.
\end{cor}

\begin{proof}
    There is a split short exact sequence \[0 \longrightarrow  M \longrightarrow M\oplus N \longrightarrow  N \longrightarrow  0\]
\end{proof}

\begin{cor}
If $R$ is noetherian then  $R^n = \bigoplus_{i\leq n}R$ is too.
\end{cor}


\begin{cor}
If $R$ is noetherian then any finitely generated module is noetherian and finitely presented
\end{cor}

\begin{proof}
Let $M$ be finitely generated, and  $\pi : R^n \longrightarrow  M$ be a surjective map. Then $M$ is a quotient of  $R^n$ which is noetherian, so  $M$ is noetherian. Then,  $\ker \pi \subset R^n$ is finitely generated so there is  $u : R^m \longrightarrow  \ker \pi$ surjective so we have an exact sequence \[
R^m \overset u\longrightarrow  R^n\overset \pi\longrightarrow M\longrightarrow 0
\] 
\end{proof}


